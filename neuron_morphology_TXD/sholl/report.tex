\documentclass[]{article}
\usepackage{lmodern}
\usepackage{amssymb,amsmath}
\usepackage{ifxetex,ifluatex}
\usepackage{fixltx2e} % provides \textsubscript
\ifnum 0\ifxetex 1\fi\ifluatex 1\fi=0 % if pdftex
  \usepackage[T1]{fontenc}
  \usepackage[utf8]{inputenc}
\else % if luatex or xelatex
  \ifxetex
    \usepackage{mathspec}
  \else
    \usepackage{fontspec}
  \fi
  \defaultfontfeatures{Ligatures=TeX,Scale=MatchLowercase}
\fi
% use upquote if available, for straight quotes in verbatim environments
\IfFileExists{upquote.sty}{\usepackage{upquote}}{}
% use microtype if available
\IfFileExists{microtype.sty}{%
\usepackage{microtype}
\UseMicrotypeSet[protrusion]{basicmath} % disable protrusion for tt fonts
}{}
\usepackage[margin=1in]{geometry}
\usepackage{hyperref}
\hypersetup{unicode=true,
            pdftitle={树突形态学分析报告},
            pdfauthor={范祖权},
            pdfborder={0 0 0},
            breaklinks=true}
\urlstyle{same}  % don't use monospace font for urls
\usepackage{graphicx,grffile}
\makeatletter
\def\maxwidth{\ifdim\Gin@nat@width>\linewidth\linewidth\else\Gin@nat@width\fi}
\def\maxheight{\ifdim\Gin@nat@height>\textheight\textheight\else\Gin@nat@height\fi}
\makeatother
% Scale images if necessary, so that they will not overflow the page
% margins by default, and it is still possible to overwrite the defaults
% using explicit options in \includegraphics[width, height, ...]{}
\setkeys{Gin}{width=\maxwidth,height=\maxheight,keepaspectratio}
\IfFileExists{parskip.sty}{%
\usepackage{parskip}
}{% else
\setlength{\parindent}{0pt}
\setlength{\parskip}{6pt plus 2pt minus 1pt}
}
\setlength{\emergencystretch}{3em}  % prevent overfull lines
\providecommand{\tightlist}{%
  \setlength{\itemsep}{0pt}\setlength{\parskip}{0pt}}
\setcounter{secnumdepth}{0}
% Redefines (sub)paragraphs to behave more like sections
\ifx\paragraph\undefined\else
\let\oldparagraph\paragraph
\renewcommand{\paragraph}[1]{\oldparagraph{#1}\mbox{}}
\fi
\ifx\subparagraph\undefined\else
\let\oldsubparagraph\subparagraph
\renewcommand{\subparagraph}[1]{\oldsubparagraph{#1}\mbox{}}
\fi

%%% Use protect on footnotes to avoid problems with footnotes in titles
\let\rmarkdownfootnote\footnote%
\def\footnote{\protect\rmarkdownfootnote}

%%% Change title format to be more compact
\usepackage{titling}

% Create subtitle command for use in maketitle
\newcommand{\subtitle}[1]{
  \posttitle{
    \begin{center}\large#1\end{center}
    }
}

\setlength{\droptitle}{-2em}
  \title{树突形态学分析报告}
  \pretitle{\vspace{\droptitle}\centering\huge}
  \posttitle{\par}
  \author{范祖权}
  \preauthor{\centering\large\emph}
  \postauthor{\par}
  \predate{\centering\large\emph}
  \postdate{\par}
  \date{2018年8月2日}


\begin{document}
\maketitle

\subsection{原始数据}

所有原始数据均为imaris软件自动导出的,文件保存路径在当前目录下的两个子文件夹中:

\begin{verbatim}
## [1] "./GFP"        "./GFP_nearby"
\end{verbatim}

每个文件夹下以神经元为单位,保存了GFP+或GFP-类的神经元形态数据。GFP+类神经元有16个样本,所以该层次下有16个子文件夹:

\begin{verbatim}
 [1] "./GFP/(GFP+)slice_1_red neuron1 statistics 20180720_Statistics"                  
 [2] "./GFP/GFP+ slice_1 20180615 neuron1 statistics 20180720_Statistics_Statistics"   
 [3] "./GFP/GFP+ slice_3 20180627 neuron1 statistics 20180720_Statistics_Statistics"   
 [4] "./GFP/GFP+ slice_4 20180627 neuron1 statistics 20180720_Statistics_Statistics"   
 [5] "./GFP/GFP+ slice_5 20180627 neuron1 statistics 20180720_Statistics_Statistics"   
 [6] "./GFP/GFP+ slice_8 20180627 neuron1 statistics 20180720_Statistics_Statistics"   
 [7] "./GFP/GFP+ slice_9 20180627 neuron1 statistics 20180720_Statistics_Statistics"   
 [8] "./GFP/new_slice_1 20x 20180518 neuron1 statistics 20180720_Statistics_Statistics"
 [9] "./GFP/new_slice_2 20x 20180518 neuron1 statistics 20180720_Statistics_Statistics"
[10] "./GFP/new_slice_3 20x 20180518 neuron1 statistics 20180720_Statistics_Statistics"
[11] "./GFP/slice_1_green neuron1 statistics 20180720_Statistics"                      
[12] "./GFP/slice_1_green neuron3 statistics 20180720_Statistics_Statistics"           
[13] "./GFP/slice_2_green neuron1 statistics 20180720_Statistics_Statistics"           
[14] "./GFP/slice_2_green neuron3 statistics 20180720_Statistics_Statistics"           
[15] "./GFP/slice_4_green neuron1 statistics 20180720_Statistics_Statistics"           
[16] "./GFP/slice_4_green neuron2 statistics 20180720_Statistics_Statistics"           
\end{verbatim}

而GFP-类神经元有10个样本:

\begin{verbatim}
 [1] "./GFP_nearby/(GFP+ near)slice_2_red_1 neuron1 statistics 20180720_Statistics_Statistics"
 [2] "./GFP_nearby/(GFP+ near)slice_3_red neuron1 statistics 20180720_Statistics_Statistics"  
 [3] "./GFP_nearby/(GFP+ near)slice_4_red neuron1 statistics 20180720_Statistics_Statistics"  
 [4] "./GFP_nearby/GFP- slice_1 20180627 neuron1 statistics 20180720_Statistics_Statistics"   
 [5] "./GFP_nearby/GFP- slice_2 20180627 neuron1 statistics 20180720_Statistics_Statistics"   
 [6] "./GFP_nearby/GFP- slice_6 20180627 neuron1 statistics 20180720_Statistics_Statistics"   
 [7] "./GFP_nearby/GFP- slice_7 20180627 neuron1 statistics 20180720_Statistics_Statistics"   
 [8] "./GFP_nearby/slice_1 20180611 neuron1 statistics 20180720_Statistics_Statistics"        
 [9] "./GFP_nearby/slice_2_red neuron1 statistics 20180720_Statistics_Statistics"             
[10] "./GFP_nearby/slice_5_red_2 neuron1 statistics 20180720_Statistics_Statistics"           
\end{verbatim}

以上每个文件夹中都含有45个csv文件,因为imaris软件对每个神经元进行了45个参数的测量。这些参数分别是:

\begin{verbatim}
 [1] "Dendrite_Area"                     
 [2] "Dendrite_Branch_Depth"             
 [3] "Dendrite_Branch_Level"             
 [4] "Dendrite_Branching_Angle"          
 [5] "Dendrite_Branching_Angle_B"        
 [6] "Dendrite_Length"                   
 [7] "Dendrite_Mean_Diameter"            
 [8] "Dendrite_No._Spines"               
 [9] "Dendrite_Orientation_Angle"        
[10] "Dendrite_Position"                 
[11] "Dendrite_Spine_Density"            
[12] "Dendrite_Straightness"             
[13] "Dendrite_Time"                     
[14] "Dendrite_Time_Index"               
[15] "Dendrite_Volume"                   
[16] "Filament_Area_(sum)"               
[17] "Filament_Dendrite_Area_(sum)"      
[18] "Filament_Dendrite_Length_(sum)"    
[19] "Filament_Dendrite_Volume_(sum)"    
[20] "Filament_Distance_from_Origin"     
[21] "Filament_Full_Branch_Depth"        
[22] "Filament_Full_Branch_Level"        
[23] "Filament_Length_(sum)"             
[24] "Filament_No._Dendrite_Branch_Pts"  
[25] "Filament_No._Dendrite_Branches"    
[26] "Filament_No._Dendrite_Segments"    
[27] "Filament_No._Dendrite_Terminal_Pts"
[28] "Filament_No._Sholl_Intersections"  
[29] "Filament_No._Spine_Branch_Pts"     
[30] "Filament_No._Spine_Segments"       
[31] "Filament_No._Spine_Terminal_Pts"   
[32] "Filament_No._Unconnected_Parts"    
[33] "Filament_Position"                 
[34] "Filament_Spine_Area_(sum)"         
[35] "Filament_Spine_Length_(sum)"       
[36] "Filament_Spine_Volume_(sum)"       
[37] "Filament_Time"                     
[38] "Filament_Time_Index"               
[39] "Filament_Volume_(sum)"             
[40] "Overall"                           
[41] "Pt_Distance_from_Origin"           
[42] "Pt_Time_Type=Dendrite"             
[43] "Pt_Time_Type=Dendrite_Beginning"   
[44] "Pt_Time_Type=Dendrite_Branch"      
[45] "Pt_Time_Type=Dendrite_Terminal"    
\end{verbatim}

每个参数都是神经元的一个指标,比如Dendrite\_Area代表树突的表面积,Dendrite\_Volume代表树突的体积,Dendrite\_Mean\_Diameter代表树突的平均直径,Dendrite\_Length代表树突的长度等等。这些参数的意义,\href{filament_statistics_preferences.html}{imaris说明文档(点击查看)}已经有详细的描述。
\#\# 数据整理
由于同一个参数的形态数据都是分散在不同文件夹中的不同文件中,需要将它们合并以便于之后的统计分析。
\href{Sum\%20of\%20Filament_No._Sholl_Intersections\%20.csv}{合并后的Sholl交点数}数据:

\begin{verbatim}
# A tibble: 10 x 11
   .id    `Filament No. Sh~ Unit  Category Radius  Time     ID Var.7 treat
   <chr>              <int> <lgl> <fct>     <dbl> <int>  <int> <lgl> <chr>
 1 (GFP+~                 0 NA    Filament     0.     1 1.00e8 NA    GFP+ 
 2 (GFP+~                 9 NA    Filament     1.     1 1.00e8 NA    GFP+ 
 3 (GFP+~                 9 NA    Filament     2.     1 1.00e8 NA    GFP+ 
 4 (GFP+~                 9 NA    Filament     3.     1 1.00e8 NA    GFP+ 
 5 (GFP+~                 9 NA    Filament     4.     1 1.00e8 NA    GFP+ 
 6 (GFP+~                11 NA    Filament     5.     1 1.00e8 NA    GFP+ 
 7 (GFP+~                 9 NA    Filament     6.     1 1.00e8 NA    GFP+ 
 8 (GFP+~                 9 NA    Filament     7.     1 1.00e8 NA    GFP+ 
 9 (GFP+~                 9 NA    Filament     8.     1 1.00e8 NA    GFP+ 
10 (GFP+~                 9 NA    Filament     9.     1 1.00e8 NA    GFP+ 
# ... with 2 more variables: `max Radius` <dbl>, `Radius Normalized` <dbl>
\end{verbatim}

.id列是每个神经元的自定义的名字,Filament No. Sholl
Intersections和Radius列分别是各个半径值对应的Sholl交点数和以1um递增的半径值。
在此基础上,也增加了计算列max Radius和Radius Normalized:max
Radius列表示最大的Sholl半径,Radius
Normalized表示以最大半径为标准的归一化值。 合并后的


\end{document}
